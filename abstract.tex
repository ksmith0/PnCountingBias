\begin{abstract}

We present a correction required in the
determination of $\beta$-delayed neutron emission emission
probabilities,
$\rm{P}_{n}$, when performed via implantation in a semiconductor.  We show that
the typical method of observing neutron emission correlated with
$\beta$-decay and implantation of a heavy ion is susceptible to an
inherent bias in the $\rm{P}_{n}$ determination due to the
implantation detector's energy detection threshold for $\beta$
particles.  The required correction which we present follows from a
simple application of the well known Bethe-Bloch equation for energy
loss of a charged massive particle in a medium.  We find a bias is
inherent to the implantation method which can cause an
overestimation of $\rm{P}_{n}$, the
$\rm{P}_{n}$-value. Using GEANT4 simulations we show that the required
correction to remove this bias can be on the order of $\sim$10\%,
which is the same order of the main reported sources of uncertainty in
current $\rm{P}_{n}$-value measurements.
\end{abstract}