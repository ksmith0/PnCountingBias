\section{Introduction}

$\beta$-delayed neutron emission, the process by which a nucleus
formed by $\beta$-decay of a neutron-rich precursor promptly emits a
neutron from an excited state, 
\begin{itemize}
\item has been observed in more than 200 nuclei~\cite{Borg13}, and is potentially experimentally accessible in 400 more~\cite{Dill14}. 
\item neutron-emission probabilities, $\rm{P}_{n}$, are of considerable interest for these 
precursors since they have important implications for the operation
of nuclear reactors and are essential to reproduce the isotopic
abundance yields of the
astrophysical rapid neutron-capture (r-) process~\cite{Gome14}. 
\item ...mention Surman,Mumpower sensitivity study
  \item Theoretical descriptions exist, but still factor of 10
  off~\cite{Krat73,Mier13}.
  \item Experimental methods typically use implantation
  +long-counter~\cite{Mehr96,Mont06}.
 \item Detection of beta-required, but this folds-in effect of detection
  threshold. Cite~\cite{Math12} pointing out this issue ...\& mention
  they say it's ok for thin Si-detectors....We investigated this and
  found that it actually can contribute to total uncertainty.
  \item Note that previous implantation
  studies~\cite{Mont06,Pere09,Hosm10} ({\bf Probably should cite
  non-NERO paper here too\dots})
  take into account beta-detection efficiency uncertainty...but not
  threshold impact on beta-detection efficiency.
  \item Note disagreement in literature between tape and DSSD Pn-values
  \item Discuss possible improper calibration of NERO efficiency and how
  this would imply a systematic shift in the Pn-value by some
  percentage (of course taking into account the neutron energies).
  \item Improper calibration is either supported by the data or not, we need to check.
  \item we propose an alternative explanation to Pn discrepancy:
	 Threshold of beta-detection biases results towards
	 overestimating Pn
\end{itemize}